%%%%%%%%%%%%%%%%%%%%%%%%%%%%%%%%%%%%%%%%%%%%%%%%%%%%%%%%%%%%%%%%%%%%%
%% This is a (brief) model paper using the achemso class
%% The document class accepts keyval options, which should include
%% the target journal and optionally the manuscript type. 
%%%%%%%%%%%%%%%%%%%%%%%%%%%%%%%%%%%%%%%%%%%%%%%%%%%%%%%%%%%%%%%%%%%%%
\documentclass[journal=jacsat,manuscript=article]{achemso}

%%%%%%%%%%%%%%%%%%%%%%%%%%%%%%%%%%%%%%%%%%%%%%%%%%%%%%%%%%%%%%%%%%%%%
%% Place any additional packages needed here.  Only include packages
%% which are essential, to avoid problems later. Do NOT use any
%% packages which require e-TeX (for example etoolbox): the e-TeX
%% extensions are not currently available on the ACS conversion
%% servers.
%%%%%%%%%%%%%%%%%%%%%%%%%%%%%%%%%%%%%%%%%%%%%%%%%%%%%%%%%%%%%%%%%%%%%
\usepackage[version=3]{mhchem} % Formula subscripts using \ce{}

%%%%%%%%%%%%%%%%%%%%%%%%%%%%%%%%%%%%%%%%%%%%%%%%%%%%%%%%%%%%%%%%%%%%%
%% If issues arise when submitting your manuscript, you may want to
%% un-comment the next line.  This provides information on the
%% version of every file you have used.
%%%%%%%%%%%%%%%%%%%%%%%%%%%%%%%%%%%%%%%%%%%%%%%%%%%%%%%%%%%%%%%%%%%%%
%%\listfiles

%%%%%%%%%%%%%%%%%%%%%%%%%%%%%%%%%%%%%%%%%%%%%%%%%%%%%%%%%%%%%%%%%%%%%
%% Place any additional macros here.  Please use \newcommand* where
%% possible, and avoid layout-changing macros (which are not used
%% when typesetting).
%%%%%%%%%%%%%%%%%%%%%%%%%%%%%%%%%%%%%%%%%%%%%%%%%%%%%%%%%%%%%%%%%%%%%
\newcommand*\mycommand[1]{\texttt{\emph{#1}}}

%%%%%%%%%%%%%%%%%%%%%%%%%%%%%%%%%%%%%%%%%%%%%%%%%%%%%%%%%%%%%%%%%%%%%
%% Meta-data block
%% ---------------
%% Each author should be given as a separate \author command.
%%
%% Corresponding authors should have an e-mail given after the author
%% name as an \email command. Phone and fax numbers can be given
%% using \phone and \fax, respectively; this information is optional.
%%
%% The affiliation of authors is given after the authors; each
%% \affiliation command applies to all preceding authors not already
%% assigned an affiliation.
%%
%% The affiliation takes an option argument for the short name.  This
%% will typically be something like "University of Somewhere".
%%
%% The \altaffiliation macro should be used for new address, etc.
%% On the other hand, \alsoaffiliation is used on a per author basis
%% when authors are associated with multiple institutions.
%%%%%%%%%%%%%%%%%%%%%%%%%%%%%%%%%%%%%%%%%%%%%%%%%%%%%%%%%%%%%%%%%%%%%
\author{Joseph P. Heindel}
\affiliation[Berkeley]
{Kenneth S. Pitzer Theory Center and Department of Chemistry, University of California, Berkeley, California94720, United States}
\alsoaffiliation[LBL]
{Chemical Sciences Division, Lawrence Berkeley National Laboratory, Berkeley, California94720, United States}
\author{Teresa Head-Gordon}
\email{thg@berkeley.edu}
\affiliation[Berkeley]
{Kenneth S. Pitzer Theory Center and Department of Chemistry, University of California, Berkeley, California94720, United States}
\alsoaffiliation[LBL]
{Chemical Sciences Division, Lawrence Berkeley National Laboratory, Berkeley, California94720, United States}
\alsoaffiliation[Berkeley2]
{Departments of Bioengineering and Chemical and Biomolecular EngineeringUniversity of California, Berkeley, California94720, United States}

%%%%%%%%%%%%%%%%%%%%%%%%%%%%%%%%%%%%%%%%%%%%%%%%%%%%%%%%%%%%%%%%%%%%%
%% The document title should be given as usual. Some journals require
%% a running title from the author: this should be supplied as an
%% optional argument to \title.
%%%%%%%%%%%%%%%%%%%%%%%%%%%%%%%%%%%%%%%%%%%%%%%%%%%%%%%%%%%%%%%%%%%%%
\title[An \textsf{achemso} demo]
  {Construction of Accurate Force Fields from Energy Decomposition Analysis: Water}

%%%%%%%%%%%%%%%%%%%%%%%%%%%%%%%%%%%%%%%%%%%%%%%%%%%%%%%%%%%%%%%%%%%%%
%% Some journals require a list of abbreviations or keywords to be
%% supplied. These should be set up here, and will be printed after
%% the title and author information, if needed.
%%%%%%%%%%%%%%%%%%%%%%%%%%%%%%%%%%%%%%%%%%%%%%%%%%%%%%%%%%%%%%%%%%%%%
\keywords{American Chemical Society, \LaTeX}

%%%%%%%%%%%%%%%%%%%%%%%%%%%%%%%%%%%%%%%%%%%%%%%%%%%%%%%%%%%%%%%%%%%%%
%% The manuscript does not need to include \maketitle, which is
%% executed automatically.
%%%%%%%%%%%%%%%%%%%%%%%%%%%%%%%%%%%%%%%%%%%%%%%%%%%%%%%%%%%%%%%%%%%%%
\begin{document}

%%%%%%%%%%%%%%%%%%%%%%%%%%%%%%%%%%%%%%%%%%%%%%%%%%%%%%%%%%%%%%%%%%%%%
%% The "tocentry" environment can be used to create an entry for the
%% graphical table of contents. It is given here as some journals
%% require that it is printed as part of the abstract page. It will
%% be automatically moved as appropriate.
%%%%%%%%%%%%%%%%%%%%%%%%%%%%%%%%%%%%%%%%%%%%%%%%%%%%%%%%%%%%%%%%%%%%%
\begin{tocentry}

Some journals require a graphical entry for the Table of Contents.
This should be laid out ``print ready'' so that the sizing of the
text is correct.

Inside the \texttt{tocentry} environment, the font used is Helvetica
8\,pt, as required by \emph{Journal of the American Chemical
Society}.

The surrounding frame is 9\,cm by 3.5\,cm, which is the maximum
permitted for  \emph{Journal of the American Chemical Society}
graphical table of content entries. The box will not resize if the
content is too big: instead it will overflow the edge of the box.

This box and the associated title will always be printed on a
separate page at the end of the document.

\end{tocentry}

%%%%%%%%%%%%%%%%%%%%%%%%%%%%%%%%%%%%%%%%%%%%%%%%%%%%%%%%%%%%%%%%%%%%%
%% The abstract environment will automatically gobble the contents
%% if an abstract is not used by the target journal.
%%%%%%%%%%%%%%%%%%%%%%%%%%%%%%%%%%%%%%%%%%%%%%%%%%%%%%%%%%%%%%%%%%%%%
\begin{abstract}
  TODO
\end{abstract}

%%%%%%%%%%%%%%%%%%%%%%%%%%%%%%%%%%%%%%%%%%%%%%%%%%%%%%%%%%%%%%%%%%%%%
%% Start the main part of the manuscript here.
%%%%%%%%%%%%%%%%%%%%%%%%%%%%%%%%%%%%%%%%%%%%%%%%%%%%%%%%%%%%%%%%%%%%%
\section{Introduction}
Historically, there have been two main approaches to including
polarization in force fields: fluctuating charges\cite{rick1994dynamical}
or induced dipoles\cite{applequist1985multipole}. There have also been
attempts to unify these approaches allowing for both charge rearrangements
and induced dipoles.\cite{stern2001combined}

Our goal in this paper is to develop a new class of polarizable
force field which is able to quantitatively reproduce all of the terms
from energy decomposition analysis (EDA). The reason we want to do this
is that by reproducing the EDA term-by-term, we can ensure that the force
field will be transferable across the phase diagram of a homogeneous
system and to new heterogenous systems.
Additionally, by reproducing the EDA term-by-term, the force field
is able to provide insights which many other models simply cannot because
they do not include particular terms. For instance, many force fields
do not include charge transfer or charge penetrations terms.
Many force fields also use terms which package the Pauli repulsion and dispersion 
energies together. We will discuss this further in the results section,
but by neglecting these terms, one limits the interpretability of the
energies and forces predicted by a force field. Even worse, one can only
exclude charge transfer and charge penetration from force fields because
these energies are strongly correlated to the Pauli repulsion (see Fig. 1).
This correlation is not guaranteed to be consistent between systems, however,
which may explain part of the difficulty in producing water models which
generalize to heterogeneous systems.

\section{Theory}
EDA splits the interaction energy into five components: Pauli repulsion,
electrostatics, dispersion, polarization, and charge transfer. The force
field described in this work will model each of these term by term, resulting
in two different models, which we will call the dipole model (DM) and
ansiotropic model (AM).

Our approach builds on the density overlap hypothesis\cite{kim1981dependence,wheatley1990overlap,gavezzotti2002calculation,van2016beyond}
which states that the short-range contributions to intermolecular
interactions is proportional to the electron density overlap. In order
for this idea to be amenable to force fields, one must use atom-centered
density overlaps. One way of doing this was developed thoroughly by
Misquitta and others\cite{misquitta2014distributed,misquitta2018isa} based
on iterated stockholder atoms which can be used to define Slater-like
densities for atoms in molecules. Since this approach has been discussed
extensivly, we will only summarize the salient points.

One can show that the overlap, $S^{ij}_\rho$, of two Slater-like atomic densities,
$\rho_i(\mathbf{r})$ and $\rho_j(\mathbf{r})$, is,
\begin{equation}
  S^{ii}_\rho=\frac{\pi D^2}{b_{ii}^3}P(b_{ii}r_{ii})\exp(-b_{ii}r_{ii})
\end{equation}
The above overlap expression is only strictly true for the exponential tail
of the Slater density and for identical atoms. The overlap between atoms
with different densities, $S^{ij}_\rho$, has a more complicated form, but
it has been shown that setting $b_{ij}=\sqrt{b_ib_j}$ allows the expression
for $S^{ii}_\rho$ to be used for different atom types with negligible
approximation\cite{van2016beyond}. The polynomial prefactor in the overlap is,
\begin{equation}
  P(b_{ij}r_{ij})=\frac13(b_{ij}r_{ij})^2 + b_{ij}r_{ij}+1
\end{equation}
where, again, we will use the combination rule $b_{ij}=\sqrt{b_ib_j}$.

\subsection*{Pauli Repulsion}
The original aim of the density overlap model was to model the
Pauli repulsion energy.\cite{wwallqvist1989new,wheatley1990overlap,gordon1996approximate}
In this work, we employ two different models for Pauli repsulison,
both inspired by the work of Van Vleet \textit{et al.}\cite{van2016beyond,van2018new}.
The first is a simple isotropic model of Pauli repulsion with energy given by,
\begin{equation}
  E_{exch} = \sum_{i<j}a_{ij}^{exch}\exp(-b_{ij}r_{ij})P(b_{ij}r_{ij})
  \label{eq:exch_iso}
\end{equation}

In Eq. \ref*{eq:exch_iso}, $a_{ij}^{exch}=a_{i}^{exch}a_{j}^{exch}$ where
$a_{i}^{exch}$ is a free parameter for atom $i$. This isotropic sum is the
approach we use for the DM model. In the anisotropic model (AM), we,
unsurprisingly, put anisotropy into the atom-specific parameters.
We do this in similarly to the approach of Van Vleet, but with some minor
modifications\cite{van2018new}. First, the anisotropy is introduced by multiplying
the isotropic term with a spherical harmonic expansion. The spherical harmonics
used here are given by,
\begin{equation}
  C_{lm}(\theta,\phi)=\sqrt{\frac{4\pi}{2l+1}}Y_{lm}(\theta,\phi)
\end{equation}
where $\theta$ and $\phi$ are the polar and azimuthal angles
calculated between two local axis systems for each atom.\cite{bernardo1994anisotropic}
That is, $\theta_i$ is the angle of the vector between two axis systems,
$\mathbf{r}_{ij}=\mathbf{r}_j-\mathbf{r}_i$, and the z-axis of axis system $i$.
$\phi_i$ is the azimuthal angle, which is the angle of $\mathbf{r}_{ij}$ in the range $[0,2\pi]$ from the
x-axis of axis system $i$.
\begin{equation}
  a_i^{exch} = a_{i,iso}^{exch}\left(1 + \sum_{l>0,k}a_{i,lk}^{exch}C_{lk}(\theta_i,\phi_i)\right)
  \label{eq:exch_params_aniso}
\end{equation}

\subsection*{Electrostatics}

We make one adjustment which is to split the electrostatic energy into
a point multipole and charge penetration contribution. We will refer to
these energies as the DMA energy and CP energy, respectively. 

\section{Parameterization}

The last important point in this model concerns the determination of the
atomic width, $b_i$. We utilize the implementation of the ISA model of
Misquitta\cite{misquitta2014distributed} available in CamCASP\cite{misquitta2016ab}
to obtain these parameters from monomer electron densities.

\section{Results and discussion}

\begin{table}[ht!]
    \begin{center}
    \begin{tabular}{lccc}
        \multicolumn{3}{c}{MAE of EDA Terms (kcal/mol)} \\\hline
        Term & Dip. Model & Aniso. Model \\\hline
        Pauli &  &  \\
        &  &  \\
        Disp. &  &  \\
        &  &  \\
        DMA Elec. &  &  \\
        &  &  \\
        CP Elec. &  &  \\
        &  &  \\
        Elec. &  &  \\
        &  &  \\
        Pol. &  &  \\
        &  &  \\
        CT &  &  \\
        &  &  \\\hline
        Total &  &  \\
        &  &  \\\hline
    \end{tabular}
    \end{center}
    \vspace{-3mm}
    \caption{\textit{Comparison of the MAE of all terms in the EDA for both the 
    dipole and anisotropic models. The first line for each term is the MAE over
    all dimers used in the parameterization. The second line is the MAE over
    all gathered data which includes dimers through pentamers. See text for explanation
    of each of the energy terms.}}
    \label{tab:mae}
\end{table}

\begin{acknowledgement}

TODO

\end{acknowledgement}

%%%%%%%%%%%%%%%%%%%%%%%%%%%%%%%%%%%%%%%%%%%%%%%%%%%%%%%%%%%%%%%%%%%%%
%% The same is true for Supporting Information, which should use the
%% suppinfo environment.
%%%%%%%%%%%%%%%%%%%%%%%%%%%%%%%%%%%%%%%%%%%%%%%%%%%%%%%%%%%%%%%%%%%%%
\begin{suppinfo}

TODO

\end{suppinfo}

%%%%%%%%%%%%%%%%%%%%%%%%%%%%%%%%%%%%%%%%%%%%%%%%%%%%%%%%%%%%%%%%%%%%%
%% The appropriate \bibliography command should be placed here.
%% Notice that the class file automatically sets \bibliographystyle
%% and also names the section correctly.
%%%%%%%%%%%%%%%%%%%%%%%%%%%%%%%%%%%%%%%%%%%%%%%%%%%%%%%%%%%%%%%%%%%%%
\bibliography{references}
\end{document}