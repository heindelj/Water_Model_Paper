%%%%%%%%%%%%%%%%%%%%%%%%%%%%%%%%%%%%%%%%%%%%%%%%%%%%%%%%%%%%%%%%%%%%%
%% This is a (brief) model paper using the achemso class
%% The document class accepts keyval options, which should include
%% the target journal and optionally the manuscript type. 
%%%%%%%%%%%%%%%%%%%%%%%%%%%%%%%%%%%%%%%%%%%%%%%%%%%%%%%%%%%%%%%%%%%%%
\documentclass[journal=jacsat,manuscript=article]{achemso}

%%%%%%%%%%%%%%%%%%%%%%%%%%%%%%%%%%%%%%%%%%%%%%%%%%%%%%%%%%%%%%%%%%%%%
%% Place any additional packages needed here.  Only include packages
%% which are essential, to avoid problems later. Do NOT use any
%% packages which require e-TeX (for example etoolbox): the e-TeX
%% extensions are not currently available on the ACS conversion
%% servers.
%%%%%%%%%%%%%%%%%%%%%%%%%%%%%%%%%%%%%%%%%%%%%%%%%%%%%%%%%%%%%%%%%%%%%
\usepackage[version=3]{mhchem} % Formula subscripts using \ce{}
\usepackage{bm}
\usepackage{xcolor}
\usepackage{dsfont}

%%%%%%%%%%%%%%%%%%%%%%%%%%%%%%%%%%%%%%%%%%%%%%%%%%%%%%%%%%%%%%%%%%%%%
%% If issues arise when submitting your manuscript, you may want to
%% un-comment the next line.  This provides information on the
%% version of every file you have used.
%%%%%%%%%%%%%%%%%%%%%%%%%%%%%%%%%%%%%%%%%%%%%%%%%%%%%%%%%%%%%%%%%%%%%
%%\listfiles

%%%%%%%%%%%%%%%%%%%%%%%%%%%%%%%%%%%%%%%%%%%%%%%%%%%%%%%%%%%%%%%%%%%%%
%% Place any additional macros here.  Please use \newcommand* where
%% possible, and avoid layout-changing macros (which are not used
%% when typesetting).
%%%%%%%%%%%%%%%%%%%%%%%%%%%%%%%%%%%%%%%%%%%%%%%%%%%%%%%%%%%%%%%%%%%%%
\newcommand*\mycommand[1]{\texttt{\emph{#1}}}

%%%%%%%%%%%%%%%%%%%%%%%%%%%%%%%%%%%%%%%%%%%%%%%%%%%%%%%%%%%%%%%%%%%%%
%% Meta-data block
%% ---------------
%% Each author should be given as a separate \author command.
%%
%% Corresponding authors should have an e-mail given after the author
%% name as an \email command. Phone and fax numbers can be given
%% using \phone and \fax, respectively; this information is optional.
%%
%% The affiliation of authors is given after the authors; each
%% \affiliation command applies to all preceding authors not already
%% assigned an affiliation.
%%
%% The affiliation takes an option argument for the short name.  This
%% will typically be something like "University of Somewhere".
%%
%% The \altaffiliation macro should be used for new address, etc.
%% On the other hand, \alsoaffiliation is used on a per author basis
%% when authors are associated with multiple institutions.
%%%%%%%%%%%%%%%%%%%%%%%%%%%%%%%%%%%%%%%%%%%%%%%%%%%%%%%%%%%%%%%%%%%%%
\author{Joseph P. Heindel}
\affiliation[Berkeley]
{Kenneth S. Pitzer Theory Center and Department of Chemistry, University of California, Berkeley, California94720, United States}
\alsoaffiliation[LBL]
{Chemical Sciences Division, Lawrence Berkeley National Laboratory, Berkeley, California94720, United States}
\author{Selim Sami}
\affiliation[Berkeley]
{Kenneth S. Pitzer Theory Center and Department of Chemistry, University of California, Berkeley, California94720, United States}
\author{Teresa Head-Gordon}
\email{thg@berkeley.edu}
\affiliation[Berkeley]
{Kenneth S. Pitzer Theory Center and Department of Chemistry, University of California, Berkeley, California94720, United States}
\alsoaffiliation[LBL]
{Chemical Sciences Division, Lawrence Berkeley National Laboratory, Berkeley, California94720, United States}
\alsoaffiliation[Berkeley2]
{Departments of Bioengineering and Chemical and Biomolecular EngineeringUniversity of California, Berkeley, California94720, United States}

%%%%%%%%%%%%%%%%%%%%%%%%%%%%%%%%%%%%%%%%%%%%%%%%%%%%%%%%%%%%%%%%%%%%%
%% The document title should be given as usual. Some journals require
%% a running title from the author: this should be supplied as an
%% optional argument to \title.
%%%%%%%%%%%%%%%%%%%%%%%%%%%%%%%%%%%%%%%%%%%%%%%%%%%%%%%%%%%%%%%%%%%%%
\title[An \textsf{achemso} demo]
  {Construction of Accurate Force Fields from Energy Decomposition Analysis: Water and Ions}

%%%%%%%%%%%%%%%%%%%%%%%%%%%%%%%%%%%%%%%%%%%%%%%%%%%%%%%%%%%%%%%%%%%%%
%% Some journals require a list of abbreviations or keywords to be
%% supplied. These should be set up here, and will be printed after
%% the title and author information, if needed.
%%%%%%%%%%%%%%%%%%%%%%%%%%%%%%%%%%%%%%%%%%%%%%%%%%%%%%%%%%%%%%%%%%%%%
\keywords{American Chemical Society, \LaTeX}

%%%%%%%%%%%%%%%%%%%%%%%%%%%%%%%%%%%%%%%%%%%%%%%%%%%%%%%%%%%%%%%%%%%%%
%% The manuscript does not need to include \maketitle, which is
%% executed automatically.
%%%%%%%%%%%%%%%%%%%%%%%%%%%%%%%%%%%%%%%%%%%%%%%%%%%%%%%%%%%%%%%%%%%%%
\begin{document}

%%%%%%%%%%%%%%%%%%%%%%%%%%%%%%%%%%%%%%%%%%%%%%%%%%%%%%%%%%%%%%%%%%%%%
%% The "tocentry" environment can be used to create an entry for the
%% graphical table of contents. It is given here as some journals
%% require that it is printed as part of the abstract page. It will
%% be automatically moved as appropriate.
%%%%%%%%%%%%%%%%%%%%%%%%%%%%%%%%%%%%%%%%%%%%%%%%%%%%%%%%%%%%%%%%%%%%%
\begin{tocentry}

Some journals require a graphical entry for the Table of Contents.
This should be laid out ``print ready'' so that the sizing of the
text is correct.

Inside the \texttt{tocentry} environment, the font used is Helvetica
8\,pt, as required by \emph{Journal of the American Chemical
Society}.

The surrounding frame is 9\,cm by 3.5\,cm, which is the maximum
permitted for  \emph{Journal of the American Chemical Society}
graphical table of content entries. The box will not resize if the
content is too big: instead it will overflow the edge of the box.

This box and the associated title will always be printed on a
separate page at the end of the document.

\end{tocentry}

%%%%%%%%%%%%%%%%%%%%%%%%%%%%%%%%%%%%%%%%%%%%%%%%%%%%%%%%%%%%%%%%%%%%%
%% The abstract environment will automatically gobble the contents
%% if an abstract is not used by the target journal.
%%%%%%%%%%%%%%%%%%%%%%%%%%%%%%%%%%%%%%%%%%%%%%%%%%%%%%%%%%%%%%%%%%%%%
\begin{abstract}
This work describes a force field for water and atomic ions which aims to quantitaively
reproduce each term of an energy decomposition analysis. In the model, polarization
is handled in a manner that allows for charge fluctuations and induced dipoles.
The model contains a new approach to modelling charge transfer which allows for
explicit movement of charge between molecules. We show that this approach naturally
describes many-body charge transfer by coupling into the polarization equations.
Additionally, we highlight the fact that many-body charge transfer is non-negligible in aqueous systems.
\color{red}{More here...}
\end{abstract}

%%%%%%%%%%%%%%%%%%%%%%%%%%%%%%%%%%%%%%%%%%%%%%%%%%%%%%%%%%%%%%%%%%%%%
%% Start the main part of the manuscript here.
%%%%%%%%%%%%%%%%%%%%%%%%%%%%%%%%%%%%%%%%%%%%%%%%%%%%%%%%%%%%%%%%%%%%%
\section{Introduction}
Historically, there have been two main approaches to including
polarization in force fields: fluctuating charges\cite{rick1994dynamical}
or induced dipoles\cite{applequist1985multipole}. There have also been
attempts to unify these approaches allowing for both charge rearrangements
and induced dipoles.\cite{stern2001combined}

Our goal in this paper is to develop a new class of polarizable
force field which is able to quantitatively reproduce all of the terms
in an energy decomposition analysis (EDA)\cite{horn2016probing,mao2021intermolecular}. The reason to do this
is that by reproducing the EDA term-by-term, we can ensure that the force
field will be transferable across the phase diagram of a homogeneous
system and, ideally, to new heterogenous systems.
Additionally, by reproducing the EDA term-by-term, the force field
is able to provide insights which many other models simply cannot because
they do not include particular terms. For instance, many force fields
do not include charge transfer or charge penetrations terms.
Many force fields also use terms which package the Pauli repulsion and dispersion 
energies together. We will discuss this further in the results section,
but by neglecting these terms, one limits the interpretability of the
energies and forces predicted by a force field. Even worse, one can only
exclude charge transfer and charge penetration from force fields because
these energies are strongly correlated to the Pauli repulsion (see Fig. 1).
This correlation is not guaranteed to be consistent between systems, however,
which may explain part of the difficulty in producing water models which
generalize to heterogeneous systems.

The model described here borrows ideas from many other force fields which
have proven to be successful. Foremost among these is the use of a model
electron density such as that employed in both the HIPPO\cite{rackers2021polarizable}
and the MASTIFF models\cite{van2016beyond,van2018new}. Specifically, treating
short-range interactions as being related to the overlap of a Slater density
or Slater orbital is extremely useful. Especially since many of these short-range
interactions need to be damped to prevent divergences, and having a model density
naturally generates the appropriate damping functions. We argue, however, that
polarization should be damped more strongly than the overlap of slater orbitals
implies. This follows from the observation that most damping functions take the form
of a polynomial multiplied by an exponential. Ideally, force fields should be free
of singularities, which means the polynomial in a damping function should be large
enough to control the polynomial scaling of the interaction in question.
\textcolor{red}{This should probably go somewhere else. Near the polarization stuff I guess...}

\section{Theory}
EDA splits the interaction energy into five components: Pauli repulsion,
electrostatics, dispersion, polarization, and charge transfer. The force
field described in this work will model each of these term by term.
Note we will use a convention of referring to all energy
terms in the force field with a $V$ and all energy terms from electronic structre
with an $E$.

Our approach borrows ideas from the density overlap hypothesis\cite{kim1981dependence,wheatley1990overlap,gavezzotti2002calculation,van2016beyond}
which states that the short-range contributions to intermolecular
interactions is proportional to the electron density overlap. In order
for this idea to be amenable to force fields, one must use atom-centered
density overlaps. One way of doing this was developed thoroughly by
Misquitta and others\cite{misquitta2014distributed,misquitta2018isa} based
on iterated stockholder atoms which can be used to define Slater-like
densities for atoms in molecules. Very similar ideas have been developed
by Rackers \textit{et al.}\cite{rackers2021polarizable} and we will make reference to the differences between
the various approaches as relevant. In any case, since this approach has been discussed
extensivly, we will only summarize the salient points.

The form of the charge density used in the model is,
\begin{equation}
  \rho(r)=\frac{Qb^3}{8\pi}e^{-br}+Z\delta(r)
  \label{eq:slater}
\end{equation}
\noindent
where $Q$ is the charge associated with the model electron density,
$Z$ is the effective nuclear charge of the atom, and $b$ defines the width
of the Slater density. The delta function, $\delta(r)$, means the core is
treated as a point particle.

One can show that the overlap, $S_{ii}^\rho$, of two identical Slater-like atomic densities
at different locations, $\rho_i(\bm{r}_i)$ and $\rho_i(\bm{r}_j)$, is,
\begin{equation}
  S_{ii}^\rho=\frac{\pi D^2}{b_{ii}^3}P(b_{ii}r_{ij})\exp(-b_{ii}r_{ij})
\end{equation}
The above overlap expression is only strictly true for the exponential tail
of the Slater density and for identical atoms. The overlap between atoms
with different densities, $S^{ij}_\rho$, has a more complicated form, but
it has been shown that setting $b_{ij}=\sqrt{b_ib_j}$ allows the expression
for $S_{ii}^\rho$ to be used for different atom types to a good
approximation\cite{van2016beyond}. The polynomial prefactor in the overlap is,
\begin{equation}
  P(b_{ij}r_{ij})=\frac13(b_{ij}r_{ij})^2 + b_{ij}r_{ij}+1
\end{equation}
where, again, we will use the combination rule $b_{ij}=\sqrt{b_ib_j}$.

Rackers \textit{et al.} utilize a similar idea in the HIPPO model\cite{rackers2021polarizable}
but rather than relying on density overlap, they treat the Slater function as an
orbital and are able to derive models of Pauli repulsion, charge penetration, and
even dispersion. Because HIPPO is derived from a model orbital, the damping functions
which prevent singularities in various short-range energetic contributions arise naturally.
We find the HIPPO approach to both Pauli repulsion and electrostatics to be physically principled
and utilize them here without significant modification.

We utilize both Slater density overlap and Slater orbital overlap in different terms within
the model. It is important to realize that either approach really just provides a
physically meaningful indication of what is "short-range" for a pair of atoms. Therefore,
combining the idea of overlapping Slater densities with overlapping Slater orbitals seems
natural to us.

\subsection*{Electrostatics}

Our description of electrostatics comes from a traditional
point multipole approach and a charge penetration contribution. We will refer to
these contributions as the DMA energy and CP energy, respectively.
Our working definition of charge penetration comes by taking the
classical electrostatic energy from EDA minus the interaction energy
when using Stone's distributed multipole analysis\cite{stone1981distributed,stone1985distributeda}
out to hexadecapoles on all atoms.

\begin{equation}
  E^{CP}=E^{elec}_{EDA}-E^{elec}_{DMA}
  \label{eq:cp}
\end{equation}

The advantage of this approach is it allows us to ensure that
our multipoles are not biased to compensate for error in the
description of charge penetration, which is essential to reproduce
the classical electrostatic energy in EDA.
All distributed multipole calculations were carried out
in Stone's Orient program.\cite{stone2002orient}

Charge penetration is described by treating each atom as having both a positively
charged core and negatively charged shell. We use the same set of damping functions
as derived by Rackers and Ponder which are appropriate for the Slater density in Eq. \ref{eq:slater}.\cite{rackers2021polarizable}
Very similar models have been applied successfully in MB-UCB\cite{das2019development}
based on functional forms proposed by Piquemal\cite{piquemal2003improved} and others\cite{wang2015general,rackers2017optimized}.

Considering the interactions of the collection of cores and shells, which are expanded in multipoles,
results in the following electrostatic energy expression:

\begin{equation}
  V_{elec}=\sum_{i<j}Z_iT_{ij}Z_j+Z_i\bm{T}_{ij}^{damp}\bm{M}_j+Z_j\bm{T}_{ji}^{damp}\bm{M}_i+\bm{M}_i\bm{T}_{ij}^{overlap}\bm{M}_j
  \label{eq:elec}
\end{equation}
\noindent
In eq. \ref{eq:elec}, the first term represents repulsive core-core interactions where $T_{ij}=1/r_{ij}$
with $Z_i$ the core charge on the $i$th atom. Note that this is not the nuclear charge but an effective
nuclear charge. The second and third terms describe attractive core-shell interactions where $\bm{M}_i$ is
a vector whose entries are the components of the multipoles located on that atom. We allow all atoms to have
multipoles up to the quadrupole. Note that the core-shell interactions are damped according to
\begin{equation}
  \bm{T}_{ij}^{damp}=
  \begin{bmatrix}
    1 & \nabla & \nabla^2 \\
  \end{bmatrix}\cdot
  \left(\frac{1}{r_{ij}}f_{ij}^{damp}(r_{ij})\right)
  \label{eq:T_damp}
\end{equation}
\noindent
The shell-shell interactions are damped, albeit with a different damping function,
and the corresponding interaction tensor is written as:
\begin{equation}
  \bm{T}_{ij}^{overlap}=
  \begin{bmatrix}
    1 & \nabla & \nabla^2 \\
    \nabla & \nabla^2 & \nabla^3 \\
    \nabla^2 & \nabla^3 & \nabla^4 \\
  \end{bmatrix}\cdot
  \left(\frac{1}{r_{ij}}f_{ij}^{overlap}(r_{ij})\right)
  \label{eq:T_overlap}
\end{equation}
\noindent
The damping functions $f_{ij}^{damp}(r_{ij})$ and $f_{ij}^{overlap}(r_{ij})$ 
take the following forms.

\begin{subequations}
  \begin{equation}
    f_{ij}^{damp}(r_{ij})=1-\left(1+\frac12b_{j}r_{ij}\right)e^{-b_{j}r_{ij}}
    \label{eq:damp_a}
  \end{equation}
  \begin{equation}
    f_{ij}^{overlap}(r_{ij})=1-\left(1+\frac{11}{16}b_{ij}r_{ij}+\frac{3}{16}(b_{ij}r_{ij})^2+\frac{1}{48}(b_{ij}r_{ij})^3\right)e^{-b_{ij}r_{ij}} 
    \label{eq:damp_b}
  \end{equation}
\end{subequations}
\noindent
Simply comparing the forms of Eq. \ref{eq:damp_a} with Eq. \ref{eq:damp_b} makes it clear
why the core-shell model of charge penetration works. Namely, as long as $b_j$ is similar
to $b_{ij}$, which it almost always will be, then the polynomial in Eq. \ref{eq:damp_b} will
be larger than that in Eq. \ref{eq:damp_a}. This means the overlap damping function is stronger
and hence the shell-shell repulsions will be damped more strongly than the core-shell
attractions. Therefore, at short-range the electrostatic energy will be more attractive than a
point multipole expansion of the density would suggest, which is exactly what is meant by
charge penetration.

The damping function in Eq. \ref{eq:damp_a} can be derived directly from the form
of the Slater density in Eq. \ref{eq:slater} by computing its electrostatic potential.
The damping function in Eq. \ref{eq:damp_b} can be derived from a symmetrized coulomb integral where each
density interacts with the damped potential generated by the other density.\cite{rackers2021polarizable}
Finally, it is important to note that these damping functions are the ones which apply
to charge-charge interactions and that as higher-order multipoles are considered, new damping
functions are generated alongside the gradients of $1/r_{ij}$. In other words, every interaction
tensor in the multipole expansion will be damped differently. The form of all relevant damping
functions have been derived and summarized elsewhere.\cite{rackers2021polarizable}

\section*{Polarization}

We introduce a combined fluctuating charge (FQ) and induced dipole model of electronic
polarization. There are two reasons we have pursued this combined polarization
approach. The first is quite simple: we want to reproduce all terms from EDA,
one of which is charge transfer. If charges are not allowed to vary, then one
cannot model the explicit transfer of charge between molecules. The second reason is
that atomic polarizabilities naturally contain both charge-flow and induced dipole
contributions.\cite{stone1985distributedb} Typically, the charge-flow contributions
are localized away\cite{ruth1994localization}, but our approach does not require nonlocal
charge flow polarizabilities.

The FQ contribution to our polarization model is a modification of the electronegativity
equalization model (EEM) of polarization.\cite{mortier1986electronegativity} In EEM,
the energy of a molecule is expanded to second-order as a function of charge, then,
these charges are allowed to interact. Mathematically, this takes the following form,
\begin{equation}
  V(\bm{q})=\sum_i \chi_i q_i + \frac12 \sum_i \eta_i q_i^2 + \sum_{i<j} \frac{q_i q_j}{r_{ij}}
  \label{eq:eem}
\end{equation}

In Eq. \ref{eq:eem}, $\chi_i$ represents the electronegativity of atom $i$ and
$\eta_i$ is the atomic hardness of atom $i$. The principle of electronegativity
equalization states that  at equilibrium, the electronegativity of all atoms
will become equal. This allows the charges in an atom to be determined by solving
a system of linear equations. There are several known shortcomings of EEM. The first
is the long-range transfer of charge between molecules, even at infinite distance,
which is unrealistic for the dielectric systems studied here.\cite{chen2007qtpie,chen2008unified}
Our solution to this problem is to only allow charge rearrangements within
a molecule and not between molecules. This constraint can be introduced using
Lagrange multipliers. A second limitation of EEM is that the approach in Eq. \ref{eq:eem}
gives back the total charge of atoms in a molecule. This is somewhat awkward because
we want to separate the multipolar electrostatics, which depends on the total charge,
from the polarization contribution which should only depend on the external potential
and field experienced by each atom. Hence, we drop the linear term and focus only on the
fluctuation of charges around the reference charge used for permanent electrostatics.
Another way of viewing this is that we are equalizing electronegativity around an "already
equalized" state. The change in electronegativity at each atom due to an environment is simply
the electric potential at that atom. We can then write the FQ contribution in our
model as,
\begin{equation}
  V(\delta \bm{q})=\frac12\sum_i \eta_i \delta q_i^2 + \sum_i \delta q_i V_i + \sum_{i<j}\frac{\delta q_i \delta q_j}{r_{ij}} + \sum_{\alpha}\lambda_\alpha \sum_{i\in\alpha}\delta q_{i}
  \label{eq:fq}
\end{equation}

To summarize, Eq. \ref{eq:fq} allows charges to rearrange, $\delta q_i$, in response to an external
potential, $V_i$, with a quadratic penalty determined by the atomic hardness, $\eta_i$.
The charge rearrangements are constrained to only occur between atoms such that rearranged
charges in a molecule sum to the total charge of a molecule. Therefore, the model has $N$ lagrange
multipliers, $\lambda_\alpha$, where $N$ is the number of molecules in the system. These molecules
could be water or an ion in this work. Note that all fluctuating charges (and induced
dipoles) are allowed to interact whether they are in the same molecule or not.

We also allow electric fields due to the environment to induce dipoles on all atoms.
The energy of an induced dipole in an electric field, $\bm{E}$, including mutual polarization is,
\begin{equation}
  V(\bm{\mu}^{ind})=-\frac12\sum_i \bm{\mu}_i^{ind}\cdot \bm{E}_i^{damp} + \sum_{i<j}\bm{\mu}^{ind}_i \bm{T}^{\mu\mu}_{ij}\bm{\mu}^{ind}_j
  \label{eq:induced_dipoles}
\end{equation}

In Eq. \ref{eq:induced_dipoles}, the induced dipoles at each polarizable site, $\bm{\mu}_i^{ind}$,
interact with an external electric field, $\bm{E}_i^{damp}$, and with each other via the dipole interaction tensor.
The field $\bm{E}_i^{damp}$ is the damped electric field generated by a Slater density
and $\bm{T}^{\mu\mu}_{ij}$ is the damped dipole-dipole interaction tensor which is derived from appropriate
gradients of $f_{ij}^{overlap}/r_{ij}$.

What now remains is to determine the values of $\delta \bm{q}$ and $\bm{\mu}^{ind}$
which minimize the total energy of the system. In order to do this, we take
the derivative with respect to each $\delta q_i$ and each component of each
$\bm{\mu}_i^{ind}$ and set them all equal to zero. This results in a system of linear
equations which can be written succintly as follows:


\begin{equation}
  \begin{pmatrix}
    \bm{T}^{qq} & \bm{1}_\lambda & \bm{T}^{q\mu} \\
    \bm{1}_\lambda^\dagger & 0 & 0 \\
    -\bm{T}^{\mu q} & 0 & \bm{T}^{\mu\mu} \\
  \end{pmatrix}
  \begin{pmatrix}
    \delta \bm{q} \\
    \bm{\lambda} \\
    \bm{\mu} \\
  \end{pmatrix}
  =
  \begin{pmatrix}
    -\bm{V} \\
    \bm{Q} \\
    \bm{E} \\
  \end{pmatrix}
  \label{eq:pol_mat}
\end{equation}

The solution vector in Eq. \ref{eq:pol_mat} contains the electric potential,
$\bm{V}$, the total charges of each molecule, $\bm{Q}$, and the electric
field on each atom $\bm{E}$. The matrix has several blocks containing the
charge-charge ($\bm{T}^{qq}$), charge-dipole ($\bm{T}^{q\mu}$),
dipole-charge ($\bm{T}^{\mu q}$), and dipole-dipole interaction tensors ($\bm{T}^{\mu\mu}$).
Note that the diagonal elements of $\bm{T}^{qq}$ are the atomic hardness $\eta$ and
the $3\times 3$ diagonal blocks of $\bm{T}^{\mu\mu}$ are the inverse polarizability tensor
$\bm{\alpha}_i^{-1}$. The block $\bm{1}_\lambda$ has a column for each molecule in the system.
An entry in that column is 1 if the $i$th atom is in that molecule and zero otherwise.
These blocks enforce the charge-conservation constraints for each molecule.
Finally, $\delta\bm{q}$ contains the optimally rearranged charges, $\bm{\lambda}$ contains
the Lagrange multipliers which enforce charge conservation, and $\bm{\mu}$ are the induced dipoles.

The form of the $ij$ entries of the multipole interaction tensors are as follows:
\begin{subequations}
  \begin{equation}
    T^{qq}_{ij}=f_3^{overlap}\frac{1}{r_{ij}}
    \label{eq:tensors_a}
  \end{equation}
  \begin{equation}
  \bm{T}^{q\mu}_{ij}=f_5^{overlap}\frac{-\bm{r}_{ij}}{r_{ij}^3}
    \label{eq:tensors_b}
  \end{equation}
  \begin{equation}
  \bm{T}^{\mu\mu}_{ij}=\left(f_7^{overlap}\frac{\bm{r}_{ij}\otimes\bm{r}_{ij}}{r_{ij}^5}-f_5^{overlap}\frac{\bm{1}}{r_{ij}^3}\right)
    \label{eq:tensors_c}
  \end{equation}
  \label{eq:tensors}
\end{subequations}

The interaction tensors in Eq. \ref{eq:tensors} are the usual cartesian multipole interaction
tensors, generated by successive gradients of $1/r_{ij}$ where $r_{ij}$ is the distance between
two atoms. These tensors are multiplied by the overlap damping function used in the permanent electrostatic
interactions. Note, however, that we have done something unusal. Naturally, $T^{qq}_{ij}$ would be damped
with $f_1^{overlap}$ since this is the correct damping function for the interaction of two
Slater charge densities. This would mean that each damping function is lowered by one order
in $\bm{T}^{q\mu}_{ij}$ and $\bm{T}^{\mu\mu}_{ij}$.

We have found that if we do not increase the damping orders used in mutual polarization, then
the polarization energy tends to be systematically too large. This effect is especially large for ions
which are polarizable and highly charged and hence strongly induce polarization. This modification
is not simply empirical though. Let us consider the induced dipoles, which will be roughly
proportional to the field experienced at each atom. Therefore, the scaling of the interaction between
two induced dipoles is $1/r_{ij}^3$ (the same scaling as for permanent dipoles) multiplied by 
the rate at which the polarizing field decays.
For instance, the field due to a dipole decays as $1/r_{ij}^3$ which means the
mutual induced dipole interactions will decay as $1/r_{ij}^6$ under that field. The damping
function $f_5^{overlap}$ turns out to be a fifth-order polynomial multiplied by an exponential.
Ideally, this would be at least sixth-order to control the significant contribution of dipole
fields to the magnitude of induced dipoles. This modification of mutual polarization is very important
as systems get larger such that some molecules have unusually large induced multipoles. As an aside,
the damping functions generated by the usual Thole damping procedure\cite{thole1981molecular},
are exponentials multiplied by first-, second-, and third-order polynomials
for charge-charge, charge-dipole, and dipole-dipole interactions, respectively.\cite{thole1981molecular} The small orders
of these polynomials certainly contribute to the historic difficulty of controlling
polarization between ions and water.\cite{jiao2006simulation,mason2012accurate}

Normally, the dipole polarizability is treated as a constant in polarizable force fields,
but Chung \textit{et al.} recently pointed out that the polarizability of ions
is significantly diminished in the aqueous phase.\cite{chung2022classical}
This effect is not exclusive to ions, but is simply more noticable for diffuse anions. We adopt a slightly
simplified version of the scheme suggested by Chung \textit{et al.} for making the polarizability dependent on the local
environment. For reasons of numerical stability, we damp the inverse polarizability, $\bm{\alpha}^{-1}$, as follows
\begin{equation}
  \bm{\alpha}^{-1}_i=
  R_i\begin{pmatrix}
    \alpha^{-1}_{xx,i} & 0 & 0 \\
    0 & \alpha^{-1}_{yy,i} & 0 \\
    0 & 0 & \alpha^{-1}_{zz,i} \\
  \end{pmatrix}R_i^T
    +\mathds{1}\sum_{i<j}k^{damp}_{ij}S_{ij}^{\rho}
    \label{eq:polarizability}
\end{equation}
\noindent
The first term in Eq. \ref{eq:polarizability} is a typical expression of the dipole polarizability
in the local axis frame of that atom. $\alpha_{xx,i}$ is the $xx$ component of the dipole polarizability
with other entries defined analogoulsy. $R_i$ is the rotation matrix that transforms the local
axis system of atom $i$ to the global axis system. The second term defines an environment-dependent
isotropic damping of the polarizability. In essence, we use the density overlap, $S_{ij}^\rho$,
as an indicator of if two atoms are near one another and $k_{ij}^{damp}=(k_{i}^{damp}+k_{j}^{damp})/2$
is a combination of free parameters that control the amount the polarizability will be modified
in response to a certain amount of density overlap.

Note that this modification of the polarizability describes a completely different
effect from the damping of induced electrostatics. In the case of multipolar interactions,
the damping arises from the fact that real charge densities have a finite width. The effect
modelled in Eq. \ref{eq:polarizability} is the shrinking of atoms which occurs due to
antisymmetrization of the wavefunction. The effect is most important for very diffuse atoms, such as \ce{I^-},
or for very close contacts such as the interaction of \ce{Li^+} with \ce{H2O}.

Finally, there is one more term in our polarization model which is designed to only contribute at very
short range. Specifically, we introduce another term proportional to density overlap,
\begin{equation}
  V_{pol,sr}=\sum_{i<j}a_{ij}^{sr}f_3^{TT}(x_{ij})\frac{\bar{\alpha}_i + \bar{\alpha}_j}{2}\frac{1}{r_{ij}^3}S_{ij}^{\rho}
\label{eq:pol_sr}
\end{equation}
\noindent
While Eq. \ref{eq:pol_sr} is much more empirical than the rest of the force field,
it seems capture the effect of quadrupole polarization, which is known to be important
in water\cite{herman2023accurate} but is very computationally expensive for the magnitude of the effect.
In Eq. \ref{eq:pol_sr}, $\bar{\alpha}_i$ is the mean dipole polarizability of atom $i$, $f_3^{TT}(x_{ij})$
is the third-order Tang-Toennies damping function, defined later in Eq. \ref{eq:TT}, and $a_{ij}^{sr}=a_{i}^{sr}a_{j}^{sr}$ is
the pairwise parameter fit for this term. This term tends to have a magnitude for water
that is similar to what one expect for quadrupole polarization (around 10\%), but is clearly
not a true model for quadrupole polarization since Eq. \ref{eq:pol_sr} is pairwise-additive
while quadrupole polarization would make small, nonzero many-body contributions.

\section*{Charge Transfer}
Charge transfer is the most difficult of the terms in EDA to model. This is
because there is no classical analogue for charge transfer. One common approach to
capturing charge transfer is to use a simple exponential dependent on the distance
between atoms.\cite{rackers2021polarizable} This captures the main effect which is the
short-range exponential stabilization due to charge delocalization. Unfortunately,
many-body charge transfer is non-negligible and this effect will be completely
missed when using just exponentials. Another attempt is to essentially treat
charge transfer the same way as polarization and solve a set of induced dipole equations.\cite{das2019development,wang2023general}
This has the benefit that it can capture many-body charge transfer. One drawback of this
approach is that it does not actually allow for charge to flow between molecules
and therefore misses some of the salient physics. It is also ambiguous if the induced
dipoles relevant to CT should be treated as real dipoles and allowed to interact 
with permanent and induced multipoles. Perhaps a bigger problem is that
charge transfer can be an even larger contribution than polarization, especially at short range.
This means the charge transfer energy would be even more susceptible to
polarization catastrophes than ordinary polarization.

For all of these reasons, we introduce a new approach to describing charge transfer
which is enabled by the fact we allow for explicit charge rearrangements in our
description of polarization. Our charge transfer model includes both direct and
indirect energy contributions. The direct contributions allow for energetic stabilization
associated with both forward and backward charge transfer.
\begin{subequations}
  \begin{equation}
  V^{CT}_{i\rightarrow j}=a_{i\rightarrow j}^{CT}S_{ij}^\rho
\end{equation}
\begin{equation}
  V^{CT}_{j\rightarrow i}=a_{j\rightarrow i}^{CT}S_{ij}^\rho
\end{equation}
\begin{equation}
  V^{CT}_{direct}=\sum_{i<j}V^{CT}_{i\rightarrow j}+V^{CT}_{j\rightarrow i}
\end{equation}
  \label{eq:ct_direct}
\end{subequations}

As shown in Eq. \ref{eq:ct_direct}, the forward and backward contributions to
charge transfer are directly proportional to the density overlap.
We take inspiration from perturbation theory which shows, roughly, that
the amount of charge transferred between two molecules is proportional to the
energy associated with forward and backward charge transfer.\cite{khaliullin2007unravelling,khaliullin2008analysis,khaliullin2009electron}
Therefore, we define the amount of charge transferred from $i$ to $j$, $\Delta Q^{CT}_{i\rightarrow j}$,
and from $j$ to $i$, $\Delta Q^{CT}_{j\rightarrow i}$, as
\begin{subequations}
  \begin{equation}
  \Delta Q^{CT}_{i\rightarrow j}=\frac{V^{CT}_{i\rightarrow j}}{\epsilon_{i\rightarrow j}}
\end{equation}
\begin{equation}
  \Delta Q^{CT}_{j\rightarrow i}=\frac{V^{CT}_{j\rightarrow i}}{\epsilon_{j\rightarrow i}}
\end{equation}
\end{subequations}

The proportionality constant between direct charge transfer
energy and the amount of transferred charge is written as $\epsilon_{i\rightarrow j}$
to emphasize that this proportionality is related to the difference in energy
of an occupied orbital on $i$ and an unoccupied orbital on $j$.\cite{khaliullin2007unravelling}

This approach is novel by allowing charge to explicitly move between
fragments. This is achieved by modifying the molecular charge constraints used in Eq. \ref{eq:pol_mat}.
The charge constraint for a fragment $A$ will then take the form,
\begin{equation}
  Q^{CT}_A=Q_A+\sum_{i\in A}\sum_{j\notin A}\Delta Q^{CT}_{j\rightarrow i}-\Delta Q^{CT}_{i\rightarrow j}
  \label{eq:charge_constraint}
\end{equation}
The charge constraint including charge transfer, $Q^{CT}_A$, is simply the
difference in charge transferred to atom $i$ (in $A$) and charge transferred from atom
$i$, summed over all atoms in molecule $A$. These charges will not be optimally
distributed, so they will be allowed to relax during the polarization process.
This allows us to capture the so-called "re-polarization"\cite{khaliullin2007unravelling}
effect in which orbitals relax after allowing for occupied-virtual mixing.
For example, when charge is transferred from oxygen to hydrogen in a water
dimer, the final excess charge will mostly come to rest on the oxygen in
the water with net-negative charge.

The major non-additivity in this charge transfer model comes from
the energy penalty associated with any atoms having nonzero values of $\delta q$.
This is consistent with two known effects that distinguish the stability of water clusters.
First, homodromic rings have particularly large non-additivities.\cite{xantheas2000cooperativity}
In our model, this type of ring benefits from many-body charge transfer since each molecule
can pass its excess charge along to the next molecule, resulting in a minor excess charge penalty.
Second, this model makes it clear why water molecules which accept two hydrogen bonds without
donating any hydrogen bonds are particularly unstable.\cite{kirov2008identifying} That is, charge
gets transferred out of the double acceptor molecule and every molecule gets
stuck with nonzero total charges. Figure 1(a) illustrates the additional stabilization
experienced by homodromic rings due to being able to pass along excess charge. Figure 1(b)
shows why double acceptor water molecules are particularly unstable.

\textbf{Consider moving this into some results section later on...}



There is one technical point worth noting about this model. Because the
charge transferred between fragments is proportional to the direct CT contributions,
the charge constraints depend on the distance between atoms. This means there
is a gradient contribution which multiplies the lagrange multipliers
with the gradient of $\Delta Q^{CT}_{i\rightarrow j}$ and $\Delta Q^{CT}_{j\rightarrow i}$.
This is not difficult or expensive to evaluate, but because it is a rather
unusual gradient term, we wanted to point this out clearly.

\subsection*{Pauli Repulsion}
The original aim of the density overlap model was to model the
Pauli repulsion energy.\cite{wallqvist1989new,wheatley1990overlap,gordon1996approximate}
The density overlap model of Pauli repulsion results in a formally exponential
repulsion at short-range.
However, Rackers and Ponder have made a convinving argument that the appropriate
scaling of Pauli repulsion is actually of the form $E^{exch}\propto e^{-b_{ij}r_{ij}}/r_{ij}$.\cite{rackers2019classical,rackers2021polarizable}
While the exponential is the more important contribution, the factor of $1/r_{ij}$
becomes important at short distance and allows for the Pauli repulsion energy
to be expressed as a multipole expansion.

The multipolar pauli repulsion model has been discussed extensively elsewhere\cite{rackers2019classical}
so we only recapitulate the main points. The basic idea is that Pauli repulsion energy
between a pair of atoms is proportional to $S^2/r_{ij}$ where $S$ is the pseudo-orbital
overlap. The pseudo-orbital is defined as $\sqrt{\rho}$ where $\rho$ is the density in Eq. \ref{eq:slater}.

Therefore, the Pauli repulsion energy can be written as
\begin{equation}
  V_{Pauli}=\sum_{i<j}\frac{K_{ij}^q S^2_{q}+K_{ij}^\mu S^2_{\mu}+K_{ij}^\Theta S^2_{\Theta}}{r_{ij}}
\label{eq:pauli}
\end{equation}
\noindent
In Eq. \ref{eq:pauli}, $S^2$ represents the orbital overlap squared with different contributions
from charges, dipoles, and quadrupoles. This is critically important since Pauli repulsion
turns out to be highly anisotropic. However, producing parameters for a complete multipole expansion
tends to result in overfitting when there is not a way to derive the initial multipoles
from electronic structure. Therefore, the proportionality constants $K_{ij}=K_iK_j$ are fit instead.
Since $S^2$ takes the form of a damped multipole expansion\cite{rackers2019classical}, these proportionality
constants mean multipoles which handle repulsion are proportional to the actual electrostatic
multipoles. One nice thing about this approach is that the calculation of electrostatics and
multipolar Pauli repulsion differs only in the choice of damping function. Therefore, significant
computation can be shared between the two terms.

The expansion of Pauli repulsion in terms of multipoles has an interesting physical interpretation.
Namely, as two electron densities begin to overlap, the electrons will be expelled
from the internuclear region in order to keep the total system wavefunction antisymmetric.
This results in a "hole" in the electron density where nuclei are exposed to one another.
In a sense, then, these multipoles describe the magnitude and shape of the depletion of electron density
between two atoms which are near one another.

\section*{Dispersion}

The dispersion energy is the simplest term in the model. We use a damped
polynomial interaction given by,
\begin{equation}
  V_{disp}=\sum_{i<j}f_6^{TT}(x_{ij})\frac{C_{6,ij}}{r_{ij}^6}
  \label{eq:disp}
\end{equation}
\noindent
In Eq. \ref{eq:disp}, $C_{6,ij}$ is the dispersion coefficient between atoms
$i$ and $j$ which is determined as $C_{6,ij}=\sqrt{C_{6,i}C_{6,j}}$ and $C_{6,i}$ is
a parameter fit to the EDA dispersion energy. $f_6^{TT}(x_{ij})$ is the sixth-order
Tang-Toennies damping function\cite{tang1984improved} which was originally derived
to damp short-range dispersion,

\begin{equation}
  f_n^{TT}(x_{ij}) = 1-e^{-x_{ij}}\sum_{k=0}^n\frac{x_{ij}^k}{k!}
  \label{eq:TT}
\end{equation}

The appropriate form of $x$ for the tail of a Slater electron density
has been derived before\cite{van2016beyond} and takes the form,
\begin{equation}
  x_{ij}=b_{ij}r_{ij}-\frac{2b_{ij}^2r_{ij}^2+3b_{ij}r_{ij}}{b_{ij}^2r_{ij}^2+3b_{ij}r_{ij}+3}
  \label{eq:TT_x}
\end{equation}

Note that the TT damping functions, Eq. \ref{eq:TT}, depend parametrically on
the choice of integer $n$. In their original work, Tang and Toennies show that the appropriate choice
of $n$ for dispersion is $n=6$. This makes the damping function an exponential multiplied by a sixth order polynomial.
This polynomial is able to control the $r^{-6}$ scaling of dispersion, while the exponential ensures no
damping at long distances. As an aside, one could also use TT damping functions of different orders to control
mutual polarization. We have tested this and it works just as well as the procedure we
described of increasing the order of $f^{overlap}$.

\section*{Reference Data}

Our model is parameterized using water clusters of size \ce{(H_2O)_n} with n=2-5.
We use 2400 dimers, trimers, tetramers, and pentamers extracted from various minimized
cluster geometries. We additionally generated 4800 psuedo-random water dimers
based on a Sobol sequence. We follow exactly the same procedure as described elsewhere.\cite{misquitta2008first}
Using the same procedure we generated 4800 ion-water dimer geometries for all ion species considered in this study,
namely \ce{F^-}, \ce{Cl^-}, \ce{Br^-}, \ce{I^-}, \ce{Li^+}, \ce{Na^+}, \ce{K^+}, \ce{Rb^+}, \ce{Cs^+}, \ce{Mg^{2+}}, and \ce{Ca^{2+}}.
For all ions, we also ran a 10ps \textit{ab initio} molecular dynamics simulation
at 500K with $\omega$B97X-V/def2-TZVPPD to generate more probable ion-water configurations. We then sampled 2400 evenly spaced
configurations from this trajectory to be used for parameterization.

Some larger ion-water clusters were also generated by the following procedure.
We used the Crest software package\cite{pracht2020automated} which uses the semi-empirical GFN2-XTB\cite{bannwarth2019gfn2}
method to search for global minima on a potential energy surface. We carried out the Crest global minimum
search with five different seed structures generated by taking water clusters, \ce{(H_2O)_n}, n=6-17, from a
water cluster database\cite{rakshit2019atlas} and replacing one water randomly with one of the ions mentioned.
We then took the structures of up to the ten lowest energy minima which had different hydrogen-bond
networks and optimized them at the $\omega$B97X-V/def2-TZVPPD level of theory. This resulted in a total
of 1044 unique ion-water clusters. These full clusters
are used to characterize the ion-water potentials, but we also extracted all possible dimers
and trimers from these clusters to be used in fitting of the ion force field parameters.

All energies and forces that are used in fitting parameters of the force field are computed
at the $\omega$B97X-V/def2-QZVPPD level of theory. In the cases where clusters are optimized at
$\omega$B97X-V/def2-TZVPPD, we recompute the energies of those clusters and any derived sub-structures
with $\omega$B97X-V/def2-QZVPPD.

\section*{Parameterization}

When parameterizing this force field, we fit each term against only the EDA contribution
to that particular energy. We also try to ensure that the force field reproduces
all phsyically meaningful monomer properties including the dipole moment, dipole derivatives,
molecular polarizability, and polarizability derivatives. Optimization of parameters is done using simple gradient descent
against the root mean-square deviation (RMSD) of predicted and EDA energies.
For Pauli repulsion and electrostatics 200 random water dimers from the datasets
described above are used in fitting. For dispersion, polarization, and charge transfer, we use
200 random water dimers, trimers, tetramers, and pentamers from the datasets described above.

When parameterizing electrostatics, we optimize against two objectives. First, we ensure
that the dipole derivatives at the equilibrium geometry of water are correct (this can be achieved
nearly exactly). Second, we optimize against the distributed multipole electrostatic energy described
near Eq. \ref{eq:cp}. This helps to ensure we end up with physically meaningful multipoles.
At this point, we freeze the total charge and dipoles on each atom so that the dipole derivatives
will remain correct. Next, we fit the value of the core charges, $Z$ and electrostatic exponents, $b_{elec}$, on each atom
with respect to the total electrostatic energy from EDA. We also allow the quadrupoles to relax
against the total electrostatic energy as a form of compensation for the lack of higher-order multipoles.

Like electrostatics, polarization parameters need to be constrained to give physically meaningful
parameters. Specifically, in addition to EDA energies, we include the polarizability and polarizability derivatives
at the $\omega$B97X-V/def2-QZVPPD equilibrium geometry of water in the fitting process.
The loss function we minimize against is,
\begin{equation}
  L_{pol}=\sum_{i}^{N} \sqrt{\frac{(E_i^{FF}-E_i^{EDA})^2}{N}} + w_1||\bm{\alpha}^{FF}-\bm{\alpha}^{EDA}||+w_2||\frac{\partial\bm{\alpha}^{FF}}{\partial \bm{r}}-\frac{\partial\bm{\alpha}^{EDA}}{\partial \bm{r}}||
\label{eq:pol_loss}
\end{equation}
\noindent
In the above, the first term is the RMSD of the predicted energies, $E_i^{FF}$, from the
EDA energies $E_i^{FF}$. The second term is the frobenius norm of the difference between the
computed and predicted molecular polarizabilities, $\bm{\alpha}$. The third term is the same as
the second but for the polarizability derivatives. The weights, $w_1$ and $w_2$ are set to
1.0 and 0.01 respectively. This, in essence, forces the molecular polarizability to be
reproduced exactly while allowing for error in the polarizability derivatives which are much
more difficult to reproduce.

The charge transfer energy and dispersion energies are simply fit against the RMSD
from their EDA energies. Note that for electrostatics and Pauli repulsion, we only use dimers in
the fitting process since electrostatics is strictly pairwise-additive and Pauli repulsion is nearly so.
Dispersion, on the other hand, has a large enough many-body contribution that if only dimers
are used in the fitting, one will systematically over-estimate the dispersion energy since
many-body dispersion is usually repulsive. There are methods for including many-body dispersion,
but we have not included such terms in this model.\cite{anatole2010two,van2018new}
We may add such terms in the future.

The Pauli repulsion term is first fit against the RMSD of the corresponding EDA energy.
The repulsion parameters are then allowed to relax against the total interaction energy
for only dimers. This procedure essentially results in improved error cancellation
which we find is still necessary for a robust force field despite the effort
to create a physically meaningful force field. We will have more to say about the necessity
of error cancellation in force fields later. It should be noted that we only allow
the Pauli repulsion to optimize against dimers so that it cannot try to correct errors
in the many-body contributions. Furthermore, we will see that the Pauli repulsion energy
still ends up providing an unbiased estimate of the EDA Pauli repulsion energy.

\section*{Results and discussion}

\textbf{TODO}

\begin{table}[ht!]
    \begin{center}
    \begin{tabular}{lccc}
        \multicolumn{3}{c}{MAE of EDA Terms (kcal/mol)} \\\hline
        Term & Dip. Model & Aniso. Model \\\hline
        Pauli &  &  \\
        &  &  \\
        Disp. &  &  \\
        &  &  \\
        DMA Elec. &  &  \\
        &  &  \\
        CP Elec. &  &  \\
        &  &  \\
        Elec. &  &  \\
        &  &  \\
        Pol. &  &  \\
        &  &  \\
        CT &  &  \\
        &  &  \\\hline
        Total &  &  \\
        &  &  \\\hline
    \end{tabular}
    \end{center}
    \vspace{-3mm}
    \caption{\textit{Comparison of the MAE of all terms in the EDA for both the 
    dipole and anisotropic models. The first line for each term is the MAE over
    all dimers used in the parameterization. The second line is the MAE over
    all gathered data which includes dimers through pentamers. See text for explanation
    of each of the energy terms.}}
    \label{tab:mae}
\end{table}

\begin{figure}
  \includegraphics*[width=\textwidth]{figures/trimer_mbe_example.png}
  \caption{Two water trimers illustrating the importance of both polarization
  and charge transfer for the stability of water molecules. FRZ corresponds to
  the frozen contribution: the sum of Pauli, electrostatic, and dispersion enegies. POL is the polarization
  energy and CT is the charge transfer energy. Energies are computed at the
  $\omega$B97X-V/def2-QZVPPD level of theory. See text for discussion.}
\end{figure}

\begin{acknowledgement}

TODO

\end{acknowledgement}

%%%%%%%%%%%%%%%%%%%%%%%%%%%%%%%%%%%%%%%%%%%%%%%%%%%%%%%%%%%%%%%%%%%%%
%% The same is true for Supporting Information, which should use the
%% suppinfo environment.
%%%%%%%%%%%%%%%%%%%%%%%%%%%%%%%%%%%%%%%%%%%%%%%%%%%%%%%%%%%%%%%%%%%%%
\begin{suppinfo}

TODO

\end{suppinfo}

%%%%%%%%%%%%%%%%%%%%%%%%%%%%%%%%%%%%%%%%%%%%%%%%%%%%%%%%%%%%%%%%%%%%%
%% The appropriate \bibliography command should be placed here.
%% Notice that the class file automatically sets \bibliographystyle
%% and also names the section correctly.
%%%%%%%%%%%%%%%%%%%%%%%%%%%%%%%%%%%%%%%%%%%%%%%%%%%%%%%%%%%%%%%%%%%%%
\bibliography{references}
\end{document}